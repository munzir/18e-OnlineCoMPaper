%% Pose Generation Algorithm for Online COM Project
%% Author: Akash Patel

%% Doc Prep
\documentclass[letterpaper, 10pt, conference]{ieeeconf}
\IEEEoverridecommandlockouts


%% Packages
\usepackage{amsmath}
\usepackage{amssymb}
\usepackage{algorithm,algorithmic}

%% Command for formatting argmax
\DeclareMathOperator*{\argmax}{argmax}

%% Algorithm Package (removes the first and last row line separators
%\floatstyle{plaintop}
%\restylefloat{algorithm}

%% Title
\title{Pose Generation Algorithm}
\author{Akash Patel\\ Krang Lab Group}
\date{July 2018}

\begin{document}

\maketitle
\thispagestyle{empty}
\pagestyle{empty}

\begin{abstract}

//TODO
// Okay lets just brainstorm here what we need to write
// 18a - Create Poses
// 18c - Param Conv
// 18f - Traj Planning

\end{abstract}

\section{Purpose}

The goal of this algorithm is to generate a good set of training poses. A "good"
set of poses is defined with the following metric. Performing a learning
algorithm (gradient descent in this case) on this set of poses should result in
convergence of all reasonable initial $\beta$s and the converged values should
have good predictions for a test set of poses.

\section{Explanation}

\subsection{Random Pose Generation}

The first step is to generate a set of poses to filter. This is done through
random sampling of the entire space, specifically the subset of the space in
which the pose is balanced and safe. \textbf{Algorithm 1} describes the
approach implemented.

\subsubsection{Balanced}
A balanced pose is defined to be a pose in which the $x_{COM}$ is close to
zero. The pose's $x_{COM}$ is determined using DART (citation needed). A
balanced pose is found by moving the qBase angle by the same angle created from the vertical axis of
the wheel axle and the $x_{COM}$ position, but in the opposite direction.

\subsubsection{Safe}
A safe pose is defined to be a pose that has no collision. This includes
self-collision as well as collision with a flat ground. Collision is implemented
using DART's collision detector using the 3D model of the robot. The collision
detector does not flag adjacent bodies in contact as collisions. To account for
conflicting adjacent bodies, the pose is passed through a filter of joint limits
for each robot joint.

\subsection{Filtering}
After we obtain a distribution of randomly sampled balanced and safe poses we
need to determine which poses from this set are "good" poses. A filtering
algorithm is devised based on discarding poses with a relatively small gradient
and only learning on the poses which result in a large gradient.

At every update step, the pose with the largest total difference of the predicted
$x_{COM}$ value ($\Phi_i \beta$) is used. This ensures that only poses with a
large enough stepsize are used and not poses in which the $\beta$s values would
change very little. The stopping condition for this filtering is either if we
run out of poses to learn on or that any learning we do results in an $x_{COM}$
of less than the $x_{COM}$ tolerance level decided (such as 2 mm).

\textbf{Algorithm 2} describes the filtering procedure.

\section{Complete Algorithm}

%% useful stuff mathbf texttt

\begin{algorithm}
    \caption{Pose Generation}
    \begin{algorithmic}[1]
        \renewcommand{\algorithmicrequire}{\textbf{Input:}}
        \renewcommand{\algorithmicensure}{\textbf{Output:}}
        \REQUIRE $n_{poses}$
        \ENSURE  $\bar{q} $ (balanced and safe set of poses)
        \STATE $j = 0$
        \WHILE {$j < n_{poses}$}
        \STATE $q_j $ = Randomly generated pose
        \STATE $\bar{q}_i =$ balanced pose of $q_j$
        \STATE //TODO Need to write the algorithm for balancing
        \IF {$\bar{q}_i$ is safe}
        \STATE Add $\bar{q}_i $ to $\bar{q}$
        \ENDIF
        \STATE $ j = j + 1 $
        \ENDWHILE
        \RETURN $\bar{q}$
    \end{algorithmic}
\end{algorithm}

\begin{algorithm}
    \caption{Pose Filtering}
    \begin{algorithmic}[1]
        \renewcommand{\algorithmicrequire}{\textbf{Input:}}
        \renewcommand{\algorithmicensure}{\textbf{Output:}}
        \REQUIRE $\bar{q} \in \mathbb{R}^{n_{DOF} \times n_{poses}}$ (input set
        of poses),
        \newline $\Phi \in \mathbb{R}^{n_{poses} \times dim(\beta)}$ ($\phi(q)$ evaluated at each given pose),
        \newline $\bar{\beta} \in \mathbb{R}^{dim(\beta) \times n_{\beta}}$ (set of initial $\beta$)
        \ENSURE  $\widetilde{q}$ (filtered set of poses)
        \STATE $ j = 0 $
        \WHILE {$j < n_{poses}$}
        \STATE $i^* = \argmax_i \sum_k | \Phi_i \beta_k |$
        \STATE $\phi^* = \Phi_{i^{*}}$
        \STATE $j = j + 1$
        \STATE $\widetilde{q}_j = \bar{q}_{i^*} $
        \STATE $\beta_k = \beta_k - \eta \cdot \phi^* \beta_k \cdot \phi^* \quad \forall \quad k \in \{1, ..., n_\beta \}$
        \IF {$ | \phi^* \beta_k | < x_{tol} \quad \forall \quad k \in \{1,
        ..., n_\beta \} $ for last few iterations}
        \STATE go to step 15
        \ELSE
        \STATE $\Phi = \Phi $ without $ \Phi_{i^*} $
        \STATE $\bar{q} = \bar{q} $ without $ \bar{q}_{i^*} $
        \ENDIF
        \ENDWHILE
        \RETURN $\widetilde{q}$
    \end{algorithmic}
\end{algorithm}

\section{RRTs}

To move from one pose to another pose, a safe method of trajectory planning is
required. We use RRTs (Rapidly Randomly Generated Trees) to accomplish this.

\end{document}
