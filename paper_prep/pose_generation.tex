%% Pose Generation Algorithm for Online COM Project
%% Author: Akash Patel

%% Doc Prep
\documentclass[letterpaper, 10pt, conference]{ieeeconf}
\IEEEoverridecommandlockouts


%% Packages
\usepackage{amsmath}
\usepackage{amssymb}
\usepackage{algorithm,algorithmic}

%% Command for formatting argmax
\DeclareMathOperator*{\argmax}{argmax}

%% Algorithm Package (removes the first and last row line separators
%\floatstyle{plaintop}
%\restylefloat{algorithm}

%% Title
\title{Pose Generation Algorithm}
\author{Akash Patel\\ Krang Lab Group}
\date{July 2018}


\begin{document}

\maketitle
\thispagestyle{empty}
\pagestyle{empty}

\begin{abstract}

//TODO

\end{abstract}

\section{Purpose}

The goal of this algorithm is to generate a good set of training poses. A "good"
set of poses is defined with the following metric. Performing a learning
algorithm (gradient descent in this case) on this set of poses should result in
convergence of all reasonable initial $\beta$s and the converged values should
have good predictions for a test set of poses.

\section{Explanation}

\subsection{Random Pose Generation}

The first step is to generate a set of poses to filter. This is done through
random sampling of the entire joint space, specifically the subset of the joint
space in which the pose is balanced and safe. \textbf{Algorithm 1} describes the
approach implemented.

\subsubsection{Balanced}
A balanced pose is defined to be a pose in which the $x_{COM}$ is close to
zero. The pose's $x_{COM}$ is determined using DART. Poses in which their
$x_{COM}$ is lower than some threshold are then considered balanced. The
threshold used in our case is 1 mm. The balanced pose of any pose is determined
using \texttt{nlopt}, an optimization library. //TODO can explain the
optimization in more detail

\subsubsection{Safe}
A safe pose is defined to be a pose that has no collision. This includes
self-collision as well as collision with a flat ground. Collision is implemented
using DART's collision detector with a 3D robot model as input. //TODO can
explain collision boxes of the model

\subsection{Filtering}
After we obtain a distribution of randomly sampled balanced and safe poses we
need to determine which poses from this set are "good" poses. A filtering
algorithm is devised based on discarding poses with a relatively small gradient and
learning on the poses which result in a large gradient.

//TODO

\textbf{Algorithm 2} describes the filtering procedure.

\section{Complete Algorithm}

%% useful stuff mathbf texttt

\begin{algorithm}
    \caption{Pose Generation}
    \begin{algorithmic}[1]
        \renewcommand{\algorithmicrequire}{\textbf{Input:}}
        \renewcommand{\algorithmicensure}{\textbf{Output:}}
        \REQUIRE $n_{poses}$
        \ENSURE  $\bar{q} $ (balanced and safe set of poses)
        \STATE $j = 0$
        \WHILE {$j < n_{poses}$}
        \STATE $q_j $ = Randomly generated pose
        \STATE $\bar{q}_i =$ balanced pose of $q_j$
        \IF {$\bar{q}_i$'s $ x_{com} <= x_{tol} $ and $\bar{q}_i$ is safe}
        \STATE Add $\bar{q}_i $ to $\bar{q}$
        \ENDIF
        \STATE $ j = j + 1 $
        \ENDWHILE
        \RETURN $\bar{q}$
    \end{algorithmic}
\end{algorithm}

\begin{algorithm}
    \caption{Pose Filtering}
    \begin{algorithmic}[1]
        \renewcommand{\algorithmicrequire}{\textbf{Input:}}
        \renewcommand{\algorithmicensure}{\textbf{Output:}}
        \REQUIRE $\bar{q} \in \mathbb{R}^{n_{DOF} \times n_{poses}}$ (input set
        of poses),
        \newline $\Phi \in \mathbb{R}^{n_{poses} \times dim(\beta)}$ ($\phi(q)$ evaluated at each given pose),
        \newline $\bar{\beta} \in \mathbb{R}^{dim(\beta) \times n_{\beta}}$ (set of initial $\beta$)
        \ENSURE  $\widetilde{q}$ (filtered set of poses)
        \STATE $ j = 0 $
        \WHILE {$j < n_{poses}$}
        \STATE $i^* = \argmax_i \sum_k | \Phi_i \beta_k |$
        \STATE $\phi^* = \Phi_{i^{*}}$
        \STATE $j = j + 1$
        \STATE $\widetilde{q}_j = \bar{q}_{i^*} $
        \STATE $\beta_k = \beta_k - \eta \cdot \phi^* \beta_k \cdot \phi^* \quad \forall \quad k \in \{1, ..., n_\beta \}$
        \IF {$ | \phi^* \beta_k | < x_{tol} \quad \forall \quad k \in \{1,
        ..., n_\beta \} $ for last few iterations}
        \STATE go to step 15
        \ELSE
        \STATE $\Phi = \Phi $ without $ \Phi_{i^*} $
        \STATE $\bar{q} = \bar{q} $ without $ \bar{q}_{i^*} $
        \ENDIF
        \ENDWHILE
        \RETURN $\bar{q}$
    \end{algorithmic}
\end{algorithm}

\section{Experiment}

//TODO

\section{Results}

//TODO
Add some good plots

\end{document}
