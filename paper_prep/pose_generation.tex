%% Pose Generation Algorithm for Online COM Project
%% Author: Akash Patel

%% Doc Prep
\documentclass[10pt, letterpaper]{article}
\usepackage[margin = 1 in]{geometry}
\usepackage[utf8]{inputenc}
\usepackage{indentfirst}
\usepackage{amsmath}
\usepackage{amssymb}

%% Title
\title{Pose Generation Algorithm}
\author{Akash Patel\\ Krang Lab Group}
\date{July 2018}


\begin{document}

\maketitle

%% Content First Part
\section{Purpose}

The goal of this algorithm is to generate a good set of training poses. A "good"
set of poses is defined with the following metric. Performing a learning
algorithm (gradient descent in this case) on this set of poses should result in
convergence of all reasonable initial $\beta$s and the converged values should
have good predictions for a test set of poses.

\section{Explanation}

\subsection{Random Pose Generation}

The first step is to generate a set of poses to filter. This is done through
random sampling of the entire joint space, specifically the subset of the joint
space in which the pose is balanced and safe.

\subsubsection{Balanced}
A balanced pose is defined to be a pose in which the $x_{COM}$ is close to
zero. The pose's $x_{COM}$ is determined using DART. Poses in which their
$x_{COM}$ is lower than some threshold are then considered balanced. The
threshold used in our case is 1 mm. The balanced pose of any pose is determined
using \texttt{nlopt}, an optimization library.

\subsubsection{Safe}
A safe pose is defined to be a pose that has no collision. This includes
self-collision as well as collision with a flat ground. Collision is implemented
using DART's collision detector with a 3D robot model as input.

\subsection{Filtering}
After we obtain a distribution of randomly sampled balanced and safe poses we
need to determine which poses from this set are "good" poses. A filtering
algorithm is used

\section{Complete Algorithm}

%% useful stuff mathbf texttt

\end{document}
